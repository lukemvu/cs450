\documentclass[a4paper,12pt]{article}

% Packages and stuff
\usepackage[top = 2.5cm, bottom = 2.5cm, left = 2.5cm, right = 2.5cm]{geometry}
\usepackage{graphicx}
\usepackage{float}
\usepackage{fancyhdr}
\usepackage{amsmath}
\usepackage{listings}
\usepackage{color}
\usepackage{listings}
 
\setlength{\parindent}{0in}

% path of graphics
\graphicspath{ {./images/} }

% Header and Footer
\pagestyle{fancy}
\fancyhf{}
% UPDATE FOR ASSIGNMENTS %
\lhead{\footnotesize Homework 4}
\rhead{\footnotesize Vu}
\cfoot{\footnotesize \thepage}


\begin{document}
	\thispagestyle{empty}
	
	\begin{tabular}{p{15.5cm}}
		{\large \bf Structure of Higher Level Languages} \\ CS450 \\
		University of Massachusetts Boston \\ Fall 2020  \\ N. Haspel\\
		\hline
		\\
	\end{tabular}

	\vspace*{0.1cm}
	
	\begin{center}
		% UPDATE FOR ASSIGNMENTS %
		{\Large \bf Homework 4}
		\vspace{1mm}
		
		{\bf November 1, 2020}
		\vspace{1mm}
		
		{\bf Luke Vu}	
	\end{center} 

	\vspace*{0.5cm}
	
%%%%%%%%%%%%%%%%%%%%%%%%%%%%%%%%%%%%%%%%	
%% Document can be filled in from here %
%%%%%%%%%%%%%%%%%%%%%%%%%%%%%%%%%%%%%%%%
\section{Part 1}
\begin{enumerate}
\item ASanswers.scm
\item (new-withdraw 25), (new-withdraw 30)
\begin{center}
\includegraphics[width=16cm]{2a}
\includegraphics[width=16cm]{2b}
\end{center}
\item ASanswers.scm
\item ASanswers.scm
\item factorial, recursive and iterative
\begin{center}
\includegraphics[width=16cm]{5a}
\includegraphics[width=16cm]{5b}
\end{center}
\end{enumerate}
\section{Part 2}
\begin{enumerate}
\item 
I was only able to complete min-cost-naive through an iterative method by tracking on an index k and the lengths of lists I was working with. Originally I started the solution trying to follow the pseudo code. I could not find a way iterate through each key of the list. I built functions to split the list, sum the weight, find min value of a list, and one to apply the procedure list-sum that attempted to return a list of computed weight-sums that my main function could pull the min from. I was trying to use the map procedure to apply this list-sum procedure onto all the values of the list but realized this was just passing the list of key weights to the function. I was hard stuck here and ended up scrapping this and coding the solution with an iterative approach. This allowed a way to traverse through the list of lists. min-cost-naive-iter takes the list, an index, and a min value. An additional base case returned the min val once all key weight pairs were iterated through. \\
\\
I gave myself a full three days to work on the assignment and this was as far as I was able to make it. I regret not just skipping over task 1 and starting the other tasks.
\newpage
\item Timed Functions\\
\lstset{basicstyle=\footnotesize\ttfamily,breaklines=true}
\begin{lstlisting}
	> (timed min-cost-naive data-small)
	(time: 0)
	142
	> (timed min-cost-naive data-medium)
	(time: 40)
	2.84
	> (timed min-cost-naive data-large)
	(time: 7390)
	4.5600000000000005
\end{lstlisting}

\item Tree Construction

\item Draw tree from medium test case.
\end{enumerate}

%\lstinputlisting{test.txt}

%\newpage
%\lstinputlisting[language=Python]{source_filename.py}
% https://en.wikibooks.org/wiki/LaTeX/Source_Code_Listings
	
\end{document}

