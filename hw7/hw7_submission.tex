\documentclass[a4paper,12pt]{article}

% Packages and stuff
\usepackage[top = 2.5cm, bottom = 2.5cm, left = 2.5cm, right = 2.5cm]{geometry}
\usepackage{graphicx}
\usepackage{float}
\usepackage{fancyhdr}
\usepackage{amsmath}
\usepackage{listings}
\usepackage{color}
\usepackage{listings}
 
\setlength{\parindent}{0in}

% path of graphics
\graphicspath{ {./images/} }

% Header and Footer
\pagestyle{fancy}
\fancyhf{}
% UPDATE FOR ASSIGNMENTS %
\lhead{\footnotesize Homework 7}
\rhead{\footnotesize Vu}
\cfoot{\footnotesize \thepage}


\begin{document}
	\thispagestyle{empty}
	
	\begin{tabular}{p{15.5cm}}
		{\large \bf Structure of Higher Level Languages} \\ CS450 \\
		University of Massachusetts Boston \\ Fall 2020  \\ N. Haspel\\
		\hline
		\\
	\end{tabular}

	\vspace*{0.1cm}
	
	\begin{center}
		% UPDATE FOR ASSIGNMENTS %
		{\Large \bf Homework 7}
		\vspace{1mm}
		
		{\bf December 8, 2020}
		\vspace{1mm}
		
		{\bf Luke Vu}	
	\end{center} 

	\vspace*{0.5cm}
	
%%%%%%%%%%%%%%%%%%%%%%%%%%%%%%%%%%%%%%%%	
%% Document can be filled in from here %
%%%%%%%%%%%%%%%%%%%%%%%%%%%%%%%%%%%%%%%%
\section{Notes}
I started with PT2 and implementing clean exits and errors. I spent a good amount of time looking at outside resources to get a better understanding of continuations. For the code here I started with \texttt{save\_continuation.scm} code for returning to the main loop, which I placed into the main (s450) loop. (s450error) prints the errr message and arguments called that caused the error then executes the return back to the main s450 loop. For (exit), I created a global variable boolean exit? that is set when (exit) is called. An if statement checks this condition at the end of the s450 loop and either calls (s450) again to loop or resets the exit variable to false and exits back to the native scheme environment. This set! on exit allows s450 to be called again and loop as normal.\\
\\
For delayed implementation, I followed some of the code in the book for thunk and delayed expression constructions and helper procedures. I ran into some issues with properly creating the thunks and detecting the delayed arguments.



%\lstinputlisting{test.txt}

%\newpage
%\lstset{basicstyle=\footnotesize\ttfamily,breaklines=true}
%\lstinputlisting[language=Python]{source_filename.py}
% https://en.wikibooks.org/wiki/LaTeX/Source_Code_Listings
	
\end{document}

